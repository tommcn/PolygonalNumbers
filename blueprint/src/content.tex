% In this file you should put the actual content of the blueprint.
% It will be used both by the web and the print version.
% It should *not* include the \begin{document}
%
% If you want to split the blueprint content into several files then
% the current file can be a simple sequence of \input. Otherwise It
% can start with a \section or \chapter for instance.

\begin{definition}[Polygonal Number]
    \label{def:Polygonal}
    \lean{Polygonal}
    \leanok
    An integer $n$ is said to be polygonal of order $m$ if:

    \[
        \exists k \in \Z \quad n = \frac{m-2}{2}\cdot (k \cdot (k-1)) + k
    \]
\end{definition}

\begin{lemma}[lemma1]
    \label{lem:lemma1}
    \lean{lemma1}
    \leanok
    Let $a$ be a non-negative real number, and $x\geq a$, then:

    \[
        \sqrt{x} - \frac{4}{\sqrt{x}} \geq \sqrt{a} - \frac{4}{\sqrt{a}}
    \]
\end{lemma}

\begin{lemma}[lemma2]
    \label{lem:lemma2}
    \lean{lemma2}
    \uses{lem:lemma1}
    \leanok
    Let $x\geq 120$, then:
    \[
        \sqrt{8\cdot x - 8} - \sqrt{6\cdot x - 3} > 4
    \]
\end{lemma}

\begin{corollary}[cor2]
    \label{cor:cor2}
    \lean{cor2}
    \leanok

    Let $x \geq 53$, then:

    \[
        \frac{5}{4} + \sqrt{8\cdot x - 4} - \sqrt{6\cdot x - \frac{15}{4}} > 4.002
    \]
\end{corollary}

\begin{corollary}[cor3]
    \label{cor:cor3}
    \lean{cor3}
    \leanok

    Let $x\geq 159$, then:

    \[
        \frac{7}{6} + \sqrt{8\cdot x + \frac{4}{9}} - \sqrt{6\cdot x - \frac{15}{4}} > 6.002
    \]
\end{corollary}

\begin{definition}[I ub]
    \label{def:I_ub}
    \lean{I_ub}
    \leanok
    
    \begin{align*}
        I_{ub} &: \Z\times \Z \to \R \\
        I_{ub} &: (n,m) \mapsto 2 \cdot \left(1 - \frac{2}{m}\right) + \sqrt{4\cdot \left(1 - \frac{2}{m}\right)^2 + 8\cdot \left(\frac{n - (m - 3)}{m}\right)}
    \end{align*}

    Where $I_{ub}$ is non-computable in Lean.
\end{definition}

\begin{definition}[I lb]
    \label{def:I_lb}
    \lean{I_lb}
    \leanok
    
    \begin{align*}
        I_{ub} &: \Z\times \Z \to \R \\
        I_{ub} &: (n,m) \mapsto 2 \cdot \left(1 - \frac{2}{m}\right) + \sqrt{4\cdot \left(1 - \frac{2}{m}\right)^2 + 8\cdot \left(\frac{n - (m - 3)}{m}\right)}
    \end{align*}

    Where $I_{lb}$ is non-computable in Lean.
\end{definition}

\begin{lemma}[lemma4]
    \label{lem:lemma4}
    \lean{lemma4}
    \leanok
    \uses{def:I_ub,def:I_lb}
    Let $n,m\in\Z$ with $m\geq 3$.

    \[
        m \geq 4 \land n \geq 53\cdot m \implies I_{ub}(n,m) - I_{lb}(n,m) > 4.002
    \]
    Further,
    \[
        m = 3\land n \geq 159\cdot m \implies I_{ub}(n,m) - I_{lb}(n,m) > 6.002
    \]

    That is, the length of the interval $I_{ub}(n,m) - I_{lb}(n,m)$ is greater than $4.002$ or $6.002$, when $m\geq 4$ or $m = 3$ respectively.
\end{lemma}

\begin{lemma}[qub]
    \label{lem:qub}
    \lean{qub}
    \leanok
    Let $p\in \R$, $c > 0$, $x\leq 0$, and $x< \frac{p}{2} + \sqrt{\left(\frac{p}{2}\right)^2 + c}$, then:

    \[
        x^2 - p \cdot x- c < 0
    \]
\end{lemma}

\begin{lemma}[qlb]
    \label{lem:qlb}
    \lean{qlb}
    \leanok
    Let $p\in \R$, $c > 0$, and $x > \frac{p}{2} + \sqrt{\left(\frac{p}{2}\right)^2 + c}$, then:

    \[
        x^2 - p \cdot x- c > 0
    \]
\end{lemma}

\begin{lemma}[I lb pos]
    \label{lem:I_lb_pos}
    \lean{I_lb_pos}
    \uses{def:I_lb}
    \leanok
    Let $n,m,b,r\in\Z$ with $0 \leq r \leq m - 3$, $b > I_{lb}(n,m)$, $3 \leq m$, $2 \cdot m \leq n$ then:

    \[
        b > 0
    \]

    i.e., $I_{lb}(n,m) > 0$ with the above assumptions.
    
\end{lemma}

\begin{lemma}[main]
    \label{lem:main}
    \lean{main}
    \uses{def:I_lb,def:I_ub}
    \leanok
    Let $n,m,b,r\in\Z$ where $b$ is odd, with $0 \leq r \leq m - 3$, $2\cdot m \leq n$, $I_{lb}(n,m)\leq b \leq I_{ub}(n,m)$, and $m \mid (n-b-r)$ then:

    \[
        a = 2\cdot \frac{n - b - r}{m} + b
    \]
    and,
    \[
       a\text{ is odd and } b^2 - 4\cdot a < 0 \text{ and } b^2 + 2 \cdot b + 4 - 3\cdot a > 0
    \]    
\end{lemma}


\begin{theorem}[mod m congr]
    \label{thm:mod_m_congr}
    \lean{mod_m_congr}
    \leanok
    Let $b_1$, $b_2$ be integers such that $b_2 = b_1 + 2$, and let $n\in \Z$, and $m\in \N$ such that $m \geq 4$. Then:
    
    \[
        \exists\ r \in \Z \text{ such that } 0 \leq r \leq m - 3\text{ and } \exists\ b\in \{b_1, b_2\} \text{ such that } n \equiv b + r \pmod{m}
    \]
\end{theorem}
    
\begin{lemma}[blist]
    \label{lemma:blist}
    \lean{blist}
    \leanok
    Let $p,q\in\R$, $k\in\N$ such that $q - p \geq 2 \cdot k$, then:

    There exists a list $b$ of $k$ integers such that:

    \[
        \exists\ m \in \Z\text{ such that } \forall\ (i \in \N, i < k), b[i] = 2 \cdot (m + i) + 1\text{ and } b[i] \leq q
    \]
\end{lemma}

\begin{lemma}[res b]
    \label{lemma:res_b}
    \lean{res_b}
    \leanok
    Let $b_1,b_2,b_3\in\Z$ such that $b_2 = b_1 + 2$, and $b_3 = b_2 + 2$, then,

    \[
        \forall n \in \Z, (3\mid n - b_1)\lor (3 \mid n - b_2) \lor (3 \mid n - b_3)
    \]
\end{lemma}

\begin{lemma}[b r]
    \label{lemma:b_r}
    \lean{b_r}
    \leanok
    Let $n,m\in\Z$, and:

    \[
        m\geq 4 \land n \geq 53 \cdot m
    \]

    or 

    \[
        m = 3 \land n \geq 159 \cdot m
    \]

    Then, $\exists\ b,r \in \Z$, where $b$ is odd, such that:

    \[
        I_{lb}(n,m) \leq b \leq I_{ub}(n,m)
    \]
    and
    \[
        0 \leq r \leq m - 3
    \]
    and
    \[
        m \mid (n - b - r)
    \]
\end{lemma}

\begin{lemma}[res b r]
    \label{lemma:res_b_r}
    \lean{res_b_r}
    \leanok

    Let $b_1,b_2\in\Z$, $b_2 = b_1 + 2$, and $n,m\in\Z$ such that $m \geq 4$, then:

    \[
        \exists\ r\in \Z\text{ such that } 0 \leq r \leq m - 3\text{ and } (m \mid (n - b_1 - r)) \lor (m \mid (n - b_2 - r))
    \]
\end{lemma}


\begin{lemma}[Cauchy's Lemma]
    \label{lem:CauchyLemma}
    \lean{CauchyLemma}
    \leanok

    Let $a,b$ be odd positive integers such that $b^2<4a$ and $3a<b^2+2b+4$, then there exists nonnegative integers $s,t,u,v$ such that:

    \[
        a=s^2+t^2+u^2+v^2 \quad \text{and} \quad b=s+t+u+v
    \]
\end{lemma}

\begin{proof}
    Omitted.
\end{proof}


%
% CPNT
%
\begin{theorem}[Cauchy's Polygonal Number Theorem]
    \label{thm:CauchyPolygonalNumberTheorem}
    \lean{CauchyPolygonalNumberTheorem}
    \uses{def:Polygonal}
    \leanok

    Let $m,n\in\N$ such that $m \geq 3$, and $n \geq 120\cdot m$ and if $m\geq 4$, $n \geq 53\cdot m$ or if $m = 3$, $n \geq 159\cdot m$. 
    
    Then $S$ is the sum of $m+1$ polygonal numbers of order $m + 2$.
\end{theorem}
