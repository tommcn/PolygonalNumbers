% In this file you should put the actual content of the blueprint.
% It will be used both by the web and the print version.
% It should *not* include the \begin{document}
%
% If you want to split the blueprint content into several files then
% the current file can be a simple sequence of \input. Otherwise It
% can start with a \section or \chapter for instance.

\begin{definition}[Polygonal Number]
    \label{def:Polygonal}
    \lean{Polygonal}
    \leanok
    An integer $n$ is said to be polygonal of order $m$ if:

    \[
        \exists k \in \Z \quad n = \frac{m-2}{2}\cdot (k \cdot (k-1)) + k
    \]
\end{definition}

% \begin{lemma}[lemma1]
%     \label{lem:lemma1}
%     \lean{lemma1}
%     \leanok
%     Let $a$ be a non-negative real number, and $x\geq a$, then:

%     \[
%         \sqrt{x} - \frac{4}{\sqrt{x}} \geq \sqrt{a} - \frac{4}{\sqrt{a}}
%     \]
% \end{lemma}

% \begin{lemma}[lemma2]
%     \label{lem:lemma2}
%     \lean{lemma2}
%     \uses{lem:lemma1}
%     \leanok
%     Let $x\geq 120$, then:
%     \[
%         \sqrt{8\cdot x - 8} - \sqrt{6\cdot x - 3} > 4
%     \]
% \end{lemma}

% \begin{corollary}[cor2]
%     \label{cor:cor2}
%     \lean{cor2}
%     \leanok

%     Let $x \geq 53$, then:

%     \[
%         \frac{5}{4} + \sqrt{8\cdot x - 4} - \sqrt{6\cdot x - \frac{15}{4}} > 4.002
%     \]
% \end{corollary}

% \begin{corollary}[cor3]
%     \label{cor:cor3}
%     \lean{cor3}
%     \leanok

%     Let $x\geq 159$, then:

%     \[
%         \frac{7}{6} + \sqrt{8\cdot x + \frac{4}{9}} - \sqrt{6\cdot x - \frac{15}{4}} > 6.002
%     \]
% \end{corollary}

% \begin{proof}
%     \leanok
%     \begin{align*}
% u(n,m) - \ell(n,m)
% & = \frac{7}{6} + 
% \sqrt{8\left(\frac{n}{m}\right) + \frac{4}{9}}
% -\sqrt{6\left(\frac{n}{m}\right) - \frac{15}{4}} - 0.002 \\
% & \geq 6
% \end{align*}
%   by Corollary~\ref{cor:3} with $x = \frac{n}{m}$.
% \end{proof}

\begin{definition}[I ub]
    \label{def:I_ub}
    \lean{I_ub}
    \leanok
    
    \begin{align*}
        I_{ub} &: \Z\times \Z \to \R \\
        I_{ub} &: (n,m) \mapsto 2 \cdot \left(1 - \frac{2}{m}\right) + \sqrt{4\cdot \left(1 - \frac{2}{m}\right)^2 + 8\cdot \left(\frac{n - (m - 3)}{m}\right)}
    \end{align*}

    Where $I_{ub}$ is non-computable in Lean.
\end{definition}

\begin{definition}[I lb]
    \label{def:I_lb}
    \lean{I_lb}
    \leanok
    
    \begin{align*}
        I_{ub} &: \Z\times \Z \to \R \\
        I_{ub} &: (n,m) \mapsto 2 \cdot \left(1 - \frac{2}{m}\right) + \sqrt{4\cdot \left(1 - \frac{2}{m}\right)^2 + 8\cdot \left(\frac{n - (m - 3)}{m}\right)}
    \end{align*}

    Where $I_{lb}$ is non-computable in Lean.
\end{definition}

%% 3.3
\begin{lemma}[Interval]
    \label{lem:interval}
    \lean{lemma4}
    \leanok
    \uses{def:I_ub,def:I_lb}
    Let $n,m\in\Z$ with $m\geq 3$.

    \[
        m \geq 4 \land n \geq 53\cdot m \implies I_{ub}(n,m) - I_{lb}(n,m) > 4.002
    \]
    Further,
    \[
        m = 3\land n \geq 159\cdot m \implies I_{ub}(n,m) - I_{lb}(n,m) > 6.002
    \]

    That is, the length of the interval $I_{ub}(n,m) - I_{lb}(n,m)$ is greater than $4.002$ or $6.002$, when $m\geq 4$ or $m = 3$ respectively.
\end{lemma}

\begin{proof}
    \leanok
    With $m \geq 4$, we have
\begin{align*}
u(n,m)-\ell(n,m)
& =
 \frac{3}{2}-\frac{1}{m}+
\sqrt{8\left(\frac{n}{m}\right)+\frac{16}{m^2}+\frac{8}{m}-4} - 
\sqrt{6\left(\frac{n}{m}\right)-\frac{3}{m}\left(1-\frac{3}{m}\right)-\frac{15}{4}} - 0.002 \\
& \geq
 \frac{3}{2}-\frac{1}{4}+
\sqrt{8\left(\frac{n}{m}\right)-4} - 
\sqrt{6\left(\frac{n}{m}\right)-\frac{15}{4}} - 0.002 \\
& =
 \frac{5}{4}+
\sqrt{8\left(\frac{n}{m}\right)-4} - 
\sqrt{6\left(\frac{n}{m}\right)-\frac{15}{4}} - 0.002 \\
& \geq 4
\end{align*}
  by Corollary 3.5 with $x = \frac{n}{m}$.
When $m = 3$, we have
\begin{align*}
u(n,m) - \ell(n,m)
& = \frac{7}{6} + 
\sqrt{8\left(\frac{n}{m}\right) + \frac{4}{9}}
-\sqrt{6\left(\frac{n}{m}\right) - \frac{15}{4}} - 0.002 \\
& \geq 6
\end{align*}
  by Corollary 3.6 with $x = \frac{n}{m}$.
\end{proof}

\begin{lemma}[qub]
    \label{lem:qub}
    \lean{qub}
    \leanok
    Let $p\in \R$, $c > 0$, $x\leq 0$, and $x< \frac{p}{2} + \sqrt{\left(\frac{p}{2}\right)^2 + c}$, then:

    \[
        x^2 - p \cdot x- c < 0
    \]
\end{lemma}

\begin{proof}
    \leanok
    Since $c > 0$, we have
    $\pm\frac{p}{2} + \sqrt{\left(\frac{p}{2}\right)^ 2 + c}
       > \pm\frac{p}{2} + \left\lvert\frac{p}{2}\right\rvert \geq 0.$
        The statement holds trivially when $x = 0$.
        Assume that $x > 0$.
        Since $x < \frac{p}{2} + \sqrt{\left(\frac{p}{2}\right)^ 2 + c}$, we have
        $x - p < -\frac{p}{2} + \sqrt{\left(\frac{p}{2}\right)^ 2 + c}$.
        Thus,
        \begin{align*}
        x^2 - px - c
        & = x(x - p) - c \\
        & < x\left(-\frac{p}{2}+\sqrt{\left(\frac{p}{2}\right)^ 2 + c})\right)-c \\
        & < \left(\frac{p}{2} + \sqrt{\left(\frac{p}{2}\right)^ 2 + c}\right)
        \left(-\frac{p}{2}+\sqrt{\left(\frac{p}{2}\right)^ 2 + c})\right) - c \\
        & = 0.
        \end{align*}
\end{proof}

\begin{lemma}[qlb]
    \label{lem:qlb}
    \lean{qlb}
    \leanok
    Let $p\in \R$, $c > 0$, and $x > \frac{p}{2} + \sqrt{\left(\frac{p}{2}\right)^2 + c}$, then:

    \[
        x^2 - p \cdot x- c > 0
    \]
\end{lemma}

\begin{proof}
    \leanok
    Since $x >\frac{p}{2} + \sqrt{\left(\frac{p}{2}\right)^ 2 + c} > 0$,
    we have $x - p > -\frac{p}{2} + \sqrt{\left(\frac{p}{2}\right)^ 2 + c}
    >-\frac{p}{2} + \sqrt{\left(\frac{p}{2}\right)^ 2 + c} > 0$. 
    Hence,
  \begin{align*}
    x^2 - px - c
    & = x(x - p) - c \\
    & > \left(\frac{p}{2} + \sqrt{\left(\frac{p}{2}\right)^ 2 + c}\right)
         (x-p) - c \\
    & > \left(\frac{p}{2} + \sqrt{\left(\frac{p}{2}\right)^ 2 + c}\right)
    \left(-\frac{p}{2}+\sqrt{\left(\frac{p}{2}\right)^ 2 + c})\right) - c \\
    & = 0.
  \end{align*}
\end{proof}

\begin{lemma}[I lb pos]
    \label{lem:I_lb_pos}
    \lean{I_lb_pos}
    \uses{def:I_lb,lem:qlb}
    \leanok
    Let $n,m,b,r\in\Z$ with $0 \leq r \leq m - 3$, $b > I_{lb}(n,m)$, $3 \leq m$, $2 \cdot m \leq n$ then:

    \[
        b > 0
    \]

    i.e., $I_{lb}(n,m) > 0$ with the above assumptions.
    
\end{lemma}

\begin{proof}
    \leanok
    Note that
    \begin{align*}
    b \geq \ell(n,m)
    & =
      \left(\frac{1}{2}-\frac{3}{m}\right)
      + \sqrt{\left(\frac{1}{2}-\frac{3}{m}\right)^2
      + 6\left(\frac{n}{m}\right) - 4}+0.001 \\ 
    & >
      \left(1-\frac{6}{m}\right)/2
      +\sqrt{\left(\left(1-\frac{6}{m}\right)/2\right)^2
      + 6\left(\frac{n-r}{m}\right) - 4} \\
    \end{align*}
    Setting $p := 1 - \frac{6}{m}$ and $c := 6 \left(\frac{n - r}{m}\right) - 4$,
      we have $c > 0$ and so, by Lemma~\ref{lem:qlb} part (b), we obtain that
    $b^2 + 2b + 4 - 3a = b ^ 2 - \left(1 - \frac{6}{m}\right) b -
    \left(6\left(\frac{n - r}{m}\right) - 4\right) > 0$.
\end{proof}

\begin{lemma}[main]
    \label{lem:main}
    \lean{main}
    \uses{def:I_lb,def:I_ub,lem:qub,lem:qlb}
    \leanok
    Let $n,m,b,r\in\Z$ where $b$ is odd, with $0 \leq r \leq m - 3$, $2\cdot m \leq n$, $I_{lb}(n,m)\leq b \leq I_{ub}(n,m)$, and $m \mid (n-b-r)$ then:
    
    \[
        a = 2\cdot \frac{n - b - r}{m} + b
        \]
        and,
        \[
            a\text{ is odd and } b^2 - 4\cdot a < 0 \text{ and } b^2 + 2 \cdot b + 4 - 3\cdot a > 0
            \]    
        \end{lemma}
        
\begin{proof}
    \leanok
Note that
\begin{align*}
b \geq \ell(n,m)
& =
  \left(\frac{1}{2}-\frac{3}{m}\right)
  + \sqrt{\left(\frac{1}{2}-\frac{3}{m}\right)^2
  + 6\left(\frac{n}{m}\right) - 4}+0.001 \\ 
& >
  \left(1-\frac{6}{m}\right)/2
  +\sqrt{\left(\left(1-\frac{6}{m}\right)/2\right)^2
  + 6\left(\frac{n-r}{m}\right) - 4} \\
\end{align*}
Setting $p := 1 - \frac{6}{m}$ and $c := 6 \left(\frac{n - r}{m}\right) - 4$,
  we have $c > 0$ and so, by Lemma~\ref{lem:qlb} part (b), we obtain that
$b^2 + 2b + 4 - 3a = b ^ 2 - \left(1 - \frac{6}{m}\right) b -
\left(6\left(\frac{n - r}{m}\right) - 4\right) > 0$.
We can also see from the above derivation that $b > 0$. 
Now, 
\begin{align*}
  b \leq u(n,m) 
  & = 2\left(1-\frac{2}{m}\right)+\sqrt{4\left(1-\frac{2}{m}\right)^2
      + 8\left(\frac{n-(m-3)}{m}\right)} - 0.001 \\
  & < \left(4\left(1-\frac{2}{m}\right)/2\right) 
      + \sqrt{\left(4\left(1-\frac{2}{m}\right)/2\right)^2
      + 8\left(\frac{n-r}{m}\right)}.
\end{align*}
Setting $p := 4 \left(1 - \frac{2}{m}\right)$ and 
$c := 8 \left(\frac{n - r}{m}\right)$,
we have $c > 0$ (as $n - r \geq 2m - (m-3) = m+3$) and so, by 
  Lemma~\ref{lem:qub} part (a),
we obtain that
$b^2 - 4a = b^2 - 4 \left(1 - \frac{2}{m}\right) b - \frac{8n - r}{m} < 0.$

\end{proof}


\begin{theorem}[mod m congr]
    \label{thm:mod_m_congr}
    \lean{mod_m_congr}
    \leanok
    Let $b_1$, $b_2$ be integers such that $b_2 = b_1 + 2$, and let $n\in \Z$, and $m\in \N$ such that $m \geq 4$. Then:
    
    \[
        \exists\ r \in \Z \text{ such that } 0 \leq r \leq m - 3\text{ and } \exists\ b\in \{b_1, b_2\} \text{ such that } n \equiv b + r \pmod{m}
    \]
\end{theorem}
    
\begin{lemma}[blist]
    \label{lem:blist}
    \lean{blist}
    \leanok
    Let $p,q\in\R$, $k\in\N$ such that $q - p \geq 2 \cdot k$, then:

    There exists a sequence $(b_i)_0^{k-1}$ of $k$ integers, and an integer $m$ such that:

    \[
        \forall\ (i = 0, \dots, k-1), b_i = 2 \cdot (m + i) + 1\land  p\leq b_i \leq q
    \]
\end{lemma}

\begin{proof}
    \leanok
    Let $\ell = \lceil p \rceil.$

    Note that $p > \ell - 1$.

    We can take $m$ to be the least integer such that $2m + 1 \geq \ell$. Indeed, for all $i = 0,\ldots,k-1$, we have that $b_i \geq b_0 = 2m + 1 \geq p$ and $b_i \leq b_{k-1} = 2(m+(k-1))+ 1 = 2m+1+2(k-1).$
    
    If $\ell$ is even, then $2m+1 = \ell + 1$.

    Hence, 
    
    \begin{align*}
        2m+1+2(k-1) &= \ell + 1 + 2(k-1) \\
        &= \ell - 1 + 2k \\
        &< p + 2k \\
        &\leq p + q - p \\
        &= q    
    \end{align*}

    If $\ell$ is odd, then $2m+1 = \ell$.
    
    Hence,
    \begin{align*}
        2m+1+2(k-1) &= \ell + 2(k-1)\\
        &= \ell - 1 + 2k - 1 \\
        &< p + 2k + 1 \\
        &\leq p + q - p - 1 \\
        &< q
    \end{align*}
\end{proof}

\begin{lemma}[res b]
    \label{lemma:res_b}
    \lean{res_b}
    \leanok
    Let $n\in\Z$, and $b_1,b_2,b_3\in\Z$ such that $b_2 = b_1 + 2$ and $b_3 = b_2 + 2$.
    Then there exists $b\in\{b_1,b_2,b_3\}$ such that:

    \[
        3 \mid n-b
    \]
\end{lemma}
\begin{proof}
    Proof by cases on $n \mod b_1$
\end{proof}


\begin{lemma}[res b r]
    \label{lemma:res_b_r}
    \lean{res_b_r}
    \leanok

    Let $b_1,b_2\in\Z$, $b_2 = b_1 + 2$, and $n,m\in\Z$ such that $m \geq 4$, then:

    \[
        \exists\ r\in \Z\text{ such that } 0 \leq r \leq m - 3\text{ and } (m \mid (n - b_1 - r)) \lor (m \mid (n - b_2 - r))
    \]
\end{lemma}

\begin{proof}
    Proof by cases on $n \mod b_1$
\end{proof}

%% Lemma 2.2
\begin{lemma}[b r]
    \label{lemma:b_r}
    \lean{b_r}
    \leanok
    \uses{lem:interval,lem:blist,def:I_ub,def:I_lb}
    Let $n,m$ be positive integers such that $m \geq 4$ and $n \geq 53\cdot m$ or if $m = 3$, $n \geq 159\cdot m$. Then there exists integers $b,r$ such that:

    \begin{enumerate}
        \item $b$ is odd
        \item $I_{lb}(n,m) \leq b \leq I_{ub}(n,m)$
        \item $0 \leq r \leq m - 3$
        \item $m \mid (n - b - r)$
    \end{enumerate}
\end{lemma}

\begin{proof}
    \leanok
    First, consider the case when $m \geq 4$ and $n \geq 53m$.
  By Lemma~\ref{lem:interval} part (a), we have $u(n,m) - \ell(n,m) \geq 4.$
  It follows from Lemma~\ref{lem:blist} that there exist odd integers
  $b_0, b_1$ in the interval $[\ell(n,m), u(n,m)]$ such that $b_1 = b_0+2$.
  Let $r'$ be the remainder when $n - b_0$ is divided by $m$.
  Note that $r' \leq m - 1$ and $n - b_0 - r' \equiv 0 \pmod{m}.$
  If $r' \geq m - 2$, set $r$ to $r'-2$ and $b$ to $b_1$.
  Since $r' \leq m-1$, we have that $r = r' - 2 \leq m - 3$.
  Also, $r = r'-2 \geq m-2-2 = m - 4 \geq 4 - 4 = 0$.
  Then setting $b$ to $b_1$, we have that
  $n - b - r = n - b_1 - (r'-2)
  = n - b_0 - r' \equiv 0 \pmod{m}$.
  Hence, $m$ divides $n - b - r$.
  Otherwise, we have $r' \leq m - 3$. Setting $r$ to $r'$ and $b$ to $b_0$,
  we have that $n - b - r = n - b_0 - r' \equiv 0 \pmod{m}.$
  Hence, $m$ divides $n - b - r$.
  Next, consider the case when $m = 3$ and $n \geq 159m$. We set $r$ to 0.
  By Lemma~\ref{lem:interval} part (b), we have $u(n,m) - \ell(n,m) \geq 6.$
  It follows from Lemma~\ref{lem:blist} that there exist odd integers $b_0,
  b_1, b_2$ in the interval $[\ell(n,m), u(n,m)]$ such that $b_1 = b_0+2$ and
  $b_2 = b_1 + 2$.
  Since $b_1 \equiv b_0 + 2 \pmod{3}$ and
  $b_2 \equiv b_1 + 2 \equiv b_0 + 4 \equiv b_0 + 1 \pmod{3}$, it follows that
  for some $b \in \{b_0, b_1,b_2\}$, we have
  $n - b - r \equiv n - b \equiv 0 \pmod{3}$.
\end{proof}


\begin{lemma}[Cauchy's Lemma]
    \label{lem:CauchyLemma}
    \lean{CauchyLemma}
    \leanok

    Let $a,b$ be odd positive integers such that $b^2<4a$ and $3a<b^2+2b+4$, then there exists nonnegative integers $s,t,u,v$ such that:

    \[
        a=s^2+t^2+u^2+v^2 \quad \text{and} \quad b=s+t+u+v
    \]
\end{lemma}

\begin{proof}
    Omitted.
\end{proof}


%
% CPNT
%
\begin{theorem}[Cauchy's Polygonal Number Theorem]
    \label{thm:CauchyPolygonalNumberTheorem}
    \lean{CauchyPolygonalNumberTheorem}
    \uses{def:Polygonal}
    \leanok

    Let $m,n\in\N$ such that $m \geq 3$, and $n \geq 120\cdot m$ and if $m\geq 4$, $n \geq 53\cdot m$ or if $m = 3$, $n \geq 159\cdot m$. 
    
    Then $S$ is the sum of $m+1$ polygonal numbers of order $m + 2$.
\end{theorem}
